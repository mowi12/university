\chapter*{1 Einleitung}
\addcontentsline{toc}{chapter}{1 Einleitung}
\setcounter{chapter}{1}
\setcounter{section}{0}
\setcounter{subsection}{0}

Die Beugung von Licht stellt ein faszinierendes Phänomen dar, das im scharfen Kontrast zur Geometrischen Optik steht, welche Licht als geradlinige Strahlen in homogenen Medien beschreibt. Während die Geometrische Optik die Reflexion und Brechung von Licht anhand von geraden Strahlen und klar definierten Gesetzen erklärt, wird die Beugung relevant, wenn Licht auf Hindernisse oder Öffnungen trifft, deren Dimensionen vergleichbar mit der Wellenlänge des Lichts sind. Im Gegensatz zur rein geometrischen Darstellung des Lichtverhaltens erfordert die Beugung eine Betrachtung von Licht als elektromagnetische Welle und basiert auf dem Prinzip der Interferenz.

Im Rahmen dieses Physikpraktikums werden wir die Phänomene der Beugung eingehend erforschen und dabei verschiedene Experimente durchführen. Von der Anwendung von Licht zur Strukturaufklärung bis zur spektralen Analyse von Licht durch Beugung an optischen Gittern werden wir die breite Palette von Beugungsexperimenten kennenlernen und verstehen. Darüber hinaus werden wir die theoretische Berechnung des Auflösungsvermögens optischer Instrumente, wie Mikroskope, unter Berücksichtigung der Beugungseffekte untersuchen.

Das Ziel dieses Praktikums ist es, ein tiefgehendes Verständnis für die Beugung von Licht zu entwickeln, ihre Anwendungen in verschiedenen optischen Experimenten zu erforschen und die Brücke zwischen theoretischen Konzepten und praktischer Umsetzung zu schlagen. Durch die Durchführung und Auswertung verschiedener Beugungsexperimente werden wir die Bedeutung der Beugung in der Physik erleben und ihre Relevanz für technologische Anwendungen in Bereichen wie Strukturaufklärung, spektrale Analyse und optische Instrumente verstehen.