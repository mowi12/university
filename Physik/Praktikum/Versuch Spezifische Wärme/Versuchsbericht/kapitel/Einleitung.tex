\chapter*{1 Einleitung}
\addcontentsline{toc}{chapter}{1 Einleitung}
\setcounter{chapter}{1}
\setcounter{section}{0}
\setcounter{subsection}{0}

Im vorliegenden Versuch steht die Bestimmung der spezifischen Wärme von verschiedenen Stoffen im Fokus. Diese Messungen erfolgen sowohl auf mechanischem Wege als auch unter Verwendung eines Thermoelements. Vor dem eigentlichen Einsatz des Thermoelements ist eine Kalibrierung erforderlich, um präzise Temperaturmessungen zu gewährleisten.

Besonderes Augenmerk gilt der Bestimmung der spezifischen Wärme von Wasser, bei der mehrere Methoden zum Einsatz kommen. Das kalibrierte Thermoelement spielt dabei eine entscheidende Rolle für genaue Temperaturmessungen. Zusätzlich werden am Ende des Experiments die Schmelzwärme von Wasser sowie die spezifische Wärme von drei verschiedenen Festkörpern ermittelt.

Diese Untersuchung kombiniert die Genauigkeit der mechanischen Messmethode mit der Vielseitigkeit des Thermoelements und ermöglicht so ein umfassendes Verständnis der thermodynamischen Eigenschaften der untersuchten Stoffe.