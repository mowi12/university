\chapter*{1 Einleitung}
\addcontentsline{toc}{chapter}{1 Einleitung}
\setcounter{chapter}{1}
\setcounter{section}{0}
\setcounter{subsection}{0}

Drehschwingungen stellen eine faszinierende Facette der physikalischen Phänomene dar und finden Anwendung in verschiedenen Bereichen, von mechanischen Systemen bis hin zur molekularen Ebene. Dieses Protokoll dokumentiert eine Reihe von Experimenten, die sich mit unterschiedlichen Aspekten von Drehschwingungen auseinandersetzen.

Der Versuchsumfang umfasst die Untersuchung freier Schwingungen, Dämpfung, erzwungener Schwingungen, Resonanz, Phasenkurven, den Energieerhaltungssatz, die Schwingungsgleichung, Regression/Fit-Analysen und Molekulardynamik. In den Experimenten werden zunächst die Eigenfrequenz und Dämpfungskonstante eines frei schwingenden Drehpendels durch die Messung von Periodendauer und Amplitudenabnahme bestimmt.

Weiterhin werden die Amplitude und Phase erzwungener Schwingungen bei verschiedenen Erregerfrequenzen gemessen und in einer Resonanzkurve aufgetragen. Durch Anpassung einer Modellfunktion an diese Kurve werden Eigenfrequenz und Dämpfungskonstante ermittelt und mit den Ergebnissen der freien Schwingung verglichen. Ein weiterer Schwerpunkt des Experiments liegt auf dem Zusammenhang zwischen Infrarotabsorption und verschiedenen Schwingungsmoden eines Moleküls, der mithilfe einer Computersimulation, genauer gesagt der 'Molekulardynamik', untersucht wird.

Diese Experimente bieten nicht nur Einblicke in grundlegende Prinzipien der Drehschwingungen, sondern demonstrieren auch die Anwendbarkeit und Relevanz dieser Konzepte in unterschiedlichen physikalischen Kontexten. Die vorliegende Untersuchung verfolgt das Ziel, die zugrundeliegenden Mechanismen von Drehschwingungen zu verstehen und ihre Bedeutung für vielfältige Anwendungen zu unterstreichen.