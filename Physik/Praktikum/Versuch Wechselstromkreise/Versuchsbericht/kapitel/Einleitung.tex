\chapter*{1 Einleitung}
\addcontentsline{toc}{chapter}{1 Einleitung}
\setcounter{chapter}{1}
\setcounter{section}{0}
\setcounter{subsection}{0}

Wie der Name des Versuchs schon sagt, geht es in diesem Versuch um Wechselstrom und desssen Eigenschaften. Dabei betrachten wir u.a. die Zweipole Widerstand, Kondensator und die Spule. Dabei werden wir ein Oszilloskop verwenden, um die Spannungsverläufe zu betrachten.\\

In der Welt der Informatik gründet sich das Fundament im Wesentlichen auf den Prinzipien der Elektronik: logische Gatter, die sämtliche Rechenoperationen verarbeiten, sowie Daten, die als Bits gespeichert und transportiert werden. Viele dieser Anwendungen operieren zwar mit Gleichstrom, der aus dem Wechselstrom durch das Netzteil gewonnen wird. Dennoch ist es in der Informatik unvermeidlich, auf Wechselspannung in Form von Rechtecksignalen zu stoßen, insbesondere im Kontext des Taktsignals.

Das Taktsignal, das zwischen den Spannungszuständen "High" und "Low" oszilliert, synchronisiert sämtliche logischen Schaltungen im Inneren eines Computers. Ebenso erfolgt die Synchronisation mit anderen Computern durch dieses Signal, wobei die beiden Rechner die aufeinanderfolgenden Flanken (den Wechsel von "High" nach "Low" oder umgekehrt) als Taktgeber nutzen.

Durch die Rechteckspannung kann jedoch, wie später erläutert wird, eine Phasenverschiebung auftreten, die je nach Ausführung zu Asynchronitäten führen kann. Um diesem entgegenzuwirken, sind viele taktabhängige Schaltungen nicht nur auf das High-Level, sondern auf die steigende, beziehungsweise fallende Flanke ausgerichtet, um einen zeitlichen Puffer zu gewährleisten. Dieser Fokus auf physikalische Prinzipien unterstreicht die fundamentale Rolle der Elektronik in der Informatik und deren Auswirkungen auf die physischen Komponenten von Rechensystemen.