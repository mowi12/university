\chapter*{2 Versuchsdurchführung und Auswertung}
\addcontentsline{toc}{chapter}{2 Versuchsdurchführung und Auswertung}
\setcounter{chapter}{2}
\setcounter{section}{0}
\setcounter{subsection}{0}

\section{Versuch 1: Strom-Spannungs Kennlinien}

    \subsection{Versuchsaufbau und -durchführung}

        Versuch 1 behandelt die Strom-Spannungs Kennlinien von Widerständen und Glühbirnen. Dazu wird der Versuchsaufbau aus Abbildung \ref{fig:versuch1} verwendet. Im ersten Teil des Versuch wird die Spannung und die Stromstärke an dem markierten Stellen gemessen und festgehalten. Im zweiten Teil des Versuchs wird die Glühbirne durch einen Ohm'schen Widerstand ersetzt und die Messung wiederholt. Der Widerstand beträgt $R = 1\ \mathrm{k\Omega}$. Diese Messwerte werden ebenfalls festgehalten. In einem $U$ über $I$ Diagramm werden die Messwerte analysiert und die Steigung um den Spannungswert $U = 0\ \mathrm{V}$ bestimmt.
        Im letzten Teil wird für jedes Wertepaar die elektrische Leistung $P$ berechnet und in einem $R$ über $P$ Diagramm aufgetragen.

        \begin{figure}[h!]
            \centering
            \includegraphics[width=0.5\textwidth]{bilder/Physik_01.png}
            \caption{Versuchsaufbau für Versuch 1}
            \label{fig:versuch1}
        \end{figure}

    \subsection{Ergebnisse}

        \begin{table}[h!]
            \centering
            \begin{tabular}{|l|l|l|l|}
                \hline
                $U$ in $\mathrm{V}$ & $I$ in $\mathrm{mA}$ & $R$ in $\mathrm{\Omega}$ & $P$ in $\mathrm{mW}$\\
                \hline\hline
                $0,00$ & $0,00$ & k.A. & $0,0$\\
                \hline
                $0,50$ & $1,40$ & $357,14$ & $0,7$\\
                \hline
                $1,00$ & $2,60$ & $184,62$ & $2,6$\\
                \hline
                $1,50$ & $3,40$ & $441,18$ & $5,1$\\
                \hline
                $2,00$ & $3,90$ & $512,82$ & $7,8$\\
                \hline
                $4,00$ & $5,70$ & $701,75$ & $22,8$\\
                \hline
                $6,00$ & $7,30$ & $821,92$ & $43,8$\\
                \hline
                $8,00$ & $8,70$ & $919,54$ & $69,6$\\
                \hline
                $10,00$ & $10,00$ & $1000,00$ & $100$\\
                \hline
                $12,00$ & $11,30$ & $1061,95$ & $135,6$\\
                \hline
                $14,00$ & $12,40$ & $1129,03$ & $173,6$\\
                \hline
                $16,00$ & $13,50$ & $1185,19$ & $216$\\
                \hline
                $18,00$ & $14,50$ & $1241,38$ & $261$\\
                \hline
                $20,00$ & $15,50$ & $1290,32$ & $310$\\
                \hline
            \end{tabular}
            \caption{Messwerte für die Glühbirne}
            \label{tab:gluehbirne}
        \end{table}

        \begin{table}[h!]
            \centering
            \begin{tabular}{|l|l|l|l|}
                \hline
                $U$ in $\mathrm{V}$ & $I$ in $\mathrm{mA}$ & $R$ in $\mathrm{\Omega}$ & $P$ in $\mathrm{mW}$\\
                \hline\hline
                $0,00$ & $0,00$ & k.A. & $0,0$\\
                \hline
                $0,50$ & $0,50$ & $1000,00$ & $0,3$\\
                \hline
                $1,00$ & $1,00$ & $1000,00$ & $1,0$\\
                \hline
                $1,50$ & $1,50$ & $1000,00$ & $2,3$\\
                \hline
                $2,00$ & $2,00$ & $1000,00$ & $4,0$\\
                \hline
                $4,00$ & $4,00$ & $1000,00$ & $16,0$\\
                \hline
                $6,00$ & $5,90$ & $1016,95$ & $35,4$\\
                \hline
                $8,00$ & $7,80$ & $1025,64$ & $62,4$\\
                \hline
                $10,00$ & $9,80$ & $1020,41$ & $98,0$\\
                \hline
                $12,00$ & $11,80$ & $1016,95$ & $141,6$\\
                \hline
                $14,00$ & $13,70$ & $1021,90$ & $191,8$\\
                \hline
                $16,00$ & $15,70$ & $1019,11$ & $251,2$\\
                \hline
                $18,00$ & $17,70$ & $1016,95$ & $318,6$\\
                \hline
                $20,00$ & $19,70$ & $1015,23$ & $394,0$\\
                \hline
            \end{tabular}
            \caption{Messwerte für den Widerstand}
            \label{tab:widerstand}
        \end{table}

        \begin{figure}[h!]
            \centering
            \includegraphics[width=0.5\textwidth]{bilder/Physik_02.png}
            \caption{$U$ über $I$ Diagramm für Glühbirne (orange) und Widerstand (grau)}
            \label{fig:wertev1}
        \end{figure}

        \begin{figure}[h!]
            \centering
            \includegraphics[width=0.5\textwidth]{bilder/Physik_03.png}
            \caption{$R$ über $P$ Diagramm für Glühbirne (orange) und Widerstand (blau)}
            \label{fig:wertev2}
        \end{figure}

        TODO: Berechnung des elektrischen Widerstands anhand der Steigung

        TODO: Diskussion der Ergebnisse

\section{Versuch 2: Schaltkreise mit Widerständen}

    \subsection{Versuchsaufbau und -durchführung}

        In Versuch 2 wird folgender Schaltkreis verwendet:

        \begin{figure}[h!]
            \centering
            \includegraphics[width=0.5\textwidth]{bilder/Physik_04.png}
            \caption{Schaltkreis für Versuch 2}
            \label{fig:versuch2}
        \end{figure}

        Zunächst wird eine Spannung $U_{\mathrm{ges}} = 5\ \mathrm{V}$ an die Schaltung angelegt. Nun werden die Teilspannung $U_{3}$ und der Teilstrom $I_{2}$ gemessen und anhand dieser Messungen die restlichen Werte berechnet. Die Messwerte werden in Tabelle \ref{tab:versuch2} festgehalten.
        Danach wird der Versuch wiederholt, jedoch wird die Spannung so gewählt, das eine Stromstärke von $I_{mathrm{ges}} = 10\ \mathrm{mA}$ fließt. Die Messwerte werden ebenfalls in Tabelle \ref{tab:versuch2} festgehalten.

    \subsection{Ergebnisse}

        \begin{table}[h!]
            \centering
            \begin{tabular}{|l|l|l|l|l|l|l|l|l|l|l|}
                \hline
                $I_{\mathrm{ges}}$ in $\mathrm{mA}$ & $I_{1}$ in $\mathrm{mA}$ & $I_{2}$ in $\mathrm{mA}$ & $I_{3}$ in $\mathrm{mA}$ & $U_{\mathrm{ges}}$ in $\mathrm{V}$ & $U_{1}$ in $\mathrm{V}$ & $U_{2}$ in $\mathrm{V}$ & $U_{3}$ in $\mathrm{V}$ & $R_{1}$ in $\mathrm{\Omega}$ & $R_{2}$ in $\mathrm{\Omega}$ & $R_{3}$ in $\mathrm{\Omega}$\\
                \hline\hline
                $3,81$ & $2,41$ & $1,40$ & $3,81$ & $5,00$ & $0,60$ & $0,60$ & $3,80$ & $248,96$ & $428,57$ & $997,38$\\
                \hline
                $10,00$ & $6,10$ & $3,90$ & $10,00$ & $13,40$ & $1,63$ & $1,63$ & $10,14$ & $267,21$ & $417,95$ & $1014,00$\\
                \hline
            \end{tabular}
            \caption{Messwerte für Versuch 2}
            \label{tab:versuch2}
        \end{table}

        Die Berechnung der Werte erfolgt bei beiden Messungen nach folgdendem Schema. Zunächst lässt sich $I_{3}$ berechnen, da die Stromstäre in einer Reihenschaltung überall gleich ist und $I_{1}$, da sich die Stromstärke in Parallelschaltungen aufteilt und $I_{\mathrm{ges}}$ und $I_{2}$ gegeben sind.
        Danach lassen sich die Spannung $U_{1}$ und $U_{2}$ berechnen, da sich die beiden Widerstände als ganzes betrachtet in Reihe geschaltet sind und sich die Spannung auf jedes Element in einer Reihenschaltung aufteilt. Und in Parallelschaltungen ist die Spannung überall gleich.
        Zuletzt lassen sich die jeweiligen Widerstände mit der Formel berechnen:

        \begin{equation}
            R = \frac{U}{I}
        \end{equation}

        TODO: Diskussion

\section{Versuch 3: Wechselspannung und Oszilloskop}

    \subsection{Versuchsaufbau und -durchführung}
        
        In Versuch 3 wird mit einem Signalgenerator eine sinusförmige Wechselspannung mit $f = 40\ \mathrm{kHz}$ erzeugt. Diese soll mit dem Oszilloskop beobachtet werden. Anschließend soll eine Schirmskizze angefertigt werden und ausgewertet werden. Die Auswertung ist in Tabelle \ref{tab:versuch3} festgehalten. Als zweite Spannung wir eine rechteckförmige Wechselspannung mit $f = 2,5\ \mathrm{kHz}$ erzeugt. Die Auswertung ist ebenfalls in Tabelle \ref{tab:versuch3} festgehalten.
        Abbildungen \ref{fig:versuch3_1} und \ref{fig:versuch3_2} zeigen die Schirmskizzen.

    \subsection{Ergebnisse}

        TODO: Schirmskizzen

        \begin{table}[h!]
            \centering
            \begin{tabular}{|l|l|l|l|}
                \hline
                $f_{\mathrm{geg}}$ in $\mathrm{kHz}$ & Amplitude in $\mathrm{V}$ & Periode $T$ in $\mathrm{\mu s}$ & Frequenz $f$ in $\mathrm{kHz}$\\
                \hline\hline
                $40,0$ & $6,2 \pm 0,4$ & $25,0 \pm 1,0$ & $40,0 \pm 1,6$\\
                \hline
                $2,5$ & $2,0 \pm 0,2$ & $400,0 \pm 20,0$ & $2,5 \pm 0,125$\\
                \hline
            \end{tabular}
            \caption{Auswertung für Versuch 3}
            \label{tab:versuch3}
        \end{table}

        Die Fehlerberechnung ergibt sich aus dem Größtfehler:

        \begin{equation}
            \Delta f = \frac{1}{T^2} \cdot \Delta \mathcal{T}
        \end{equation}

        mit $\Delta \mathcal{T} = 1\ \mathrm{\mu s}$, bzw. $\Delta \mathcal{T} = 20\ \mathrm{\mu s}$. Die Amplitude besitzt einen Fehler von $0,4\ \mathrm{V}$, bzw. $0,2\ \mathrm{V}$. Der Fehler der Amplitude ist allerdings nicht für die Berechnung der Frequenz relevant.

        TOD: Diskussion

\section{Versuch 4: Ladeverhalten eines Kondensators}

    \subsection{Versuchsaufbau und -durchführung}

        TODO: Versuchsaufbau und -durchführung
        Ohmscher Widerstand $R = 100\ \mathrm{\Omega}$

    \subsection{Ergebnisse}
        
        \begin{table}[h!]
            \centering
            \begin{tabular}{|l|l|l|l|}
                \hline
                $n$ & Zeit $t$ in $\mathrm{ms}$ & Amplitude in $V$ & $I$ in $A$\\
                \hline\hline
                $1$ & $0,000$ & $6,48$ & $0,0648$\\
                \hline
            \end{tabular}
            \caption{Messwerte für Versuch 4}
            \label{tab:versuch4}
        \end{table}

        \begin{figure}[h!]
            \centering
            \includegraphics[width=0.5\textwidth]{bilder/Physik_05.png}
            \caption{Schirmskizze für Versuch 4}
            \label{fig:versuch4}
        \end{figure}

        TODO: Berechnung von $C$
        TODO: Diskussion