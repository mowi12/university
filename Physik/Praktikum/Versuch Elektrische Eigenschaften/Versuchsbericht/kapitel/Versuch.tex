\chapter*{2 Versuchsdurchführung und Auswertung}
\addcontentsline{toc}{chapter}{2 Versuchsdurchführung und Auswertung}
\setcounter{chapter}{2}
\setcounter{section}{0}
\setcounter{subsection}{0}

\section{Versuch 1: Strom-Spannung Kennlinien}

    \subsection{Versuchsaufbau und -durchführung}

        Versuch 1 behandelt die Strom-Spannungs Kennlinien von Widerständen und Glühbirnen. Dazu wird der Versuchsaufbau aus Abbildung \ref{fig:versuch1} verwendet. Im ersten Teil des Versuch wird die Spannung und die Stromstärke an dem markierten Stellen gemessen und festgehalten. Im zweiten Teil des Versuchs wird die Glühbirne durch einen Ohmschen Widerstand ersetzt und die Messung wiederholt. Der Widerstand beträgt $R = 1\ \mathrm{k\Omega}$. Diese Messwerte werden ebenfalls festgehalten. In einem $U$ über $I$ Diagramm werden die Messwerte analysiert und die Steigung um den Spannungswert $U = 0\ \mathrm{V}$ bestimmt.
        Im letzten Teil wird für jedes Wertepaar die elektrische Leistung $P$ berechnet und in einem $R$ über $P$ Diagramm aufgetragen.

        \begin{figure}[ht!]
            \centering
            \includegraphics[width=0.5\textwidth]{bilder/Physik_01.png}
            \caption{Versuchsaufbau für Versuch 1}
            \label{fig:versuch1}
        \end{figure}
\newpage
    \subsection{Ergebnisse}

        \begin{table}[ht!]
            \centering
            \begin{tabular}{|l|l|l|l|}
                \hline
                $U$ in $\mathrm{V}$ & $I$ in $\mathrm{mA}$ & $R$ in $\mathrm{\Omega}$ & $P$ in $\mathrm{mW}$\\
                \hline\hline
                $0,00$ & $0,00$ & k.A. & $0,0$\\
                \hline
                $0,50$ & $1,40$ & $357,14$ & $0,7$\\
                \hline
                $1,00$ & $2,60$ & $184,62$ & $2,6$\\
                \hline
                $1,50$ & $3,40$ & $441,18$ & $5,1$\\
                \hline
                $2,00$ & $3,90$ & $512,82$ & $7,8$\\
                \hline
                $4,00$ & $5,70$ & $701,75$ & $22,8$\\
                \hline
                $6,00$ & $7,30$ & $821,92$ & $43,8$\\
                \hline
                $8,00$ & $8,70$ & $919,54$ & $69,6$\\
                \hline
                $10,00$ & $10,00$ & $1000,00$ & $100$\\
                \hline
                $12,00$ & $11,30$ & $1061,95$ & $135,6$\\
                \hline
                $14,00$ & $12,40$ & $1129,03$ & $173,6$\\
                \hline
                $16,00$ & $13,50$ & $1185,19$ & $216$\\
                \hline
                $18,00$ & $14,50$ & $1241,38$ & $261$\\
                \hline
                $20,00$ & $15,50$ & $1290,32$ & $310$\\
                \hline
            \end{tabular}
            \caption{Messwerte für die Glühbirne}
            \label{tab:gluehbirne}
        \end{table}

        \begin{table}[ht!]
            \centering
            \begin{tabular}{|l|l|l|l|}
                \hline
                $U$ in $\mathrm{V}$ & $I$ in $\mathrm{mA}$ & $R$ in $\mathrm{\Omega}$ & $P$ in $\mathrm{mW}$\\
                \hline\hline
                $0,00$ & $0,00$ & k.A. & $0,0$\\
                \hline
                $0,50$ & $0,50$ & $1000,00$ & $0,3$\\
                \hline
                $1,00$ & $1,00$ & $1000,00$ & $1,0$\\
                \hline
                $1,50$ & $1,50$ & $1000,00$ & $2,3$\\
                \hline
                $2,00$ & $2,00$ & $1000,00$ & $4,0$\\
                \hline
                $4,00$ & $4,00$ & $1000,00$ & $16,0$\\
                \hline
                $6,00$ & $5,90$ & $1016,95$ & $35,4$\\
                \hline
                $8,00$ & $7,80$ & $1025,64$ & $62,4$\\
                \hline
                $10,00$ & $9,80$ & $1020,41$ & $98,0$\\
                \hline
                $12,00$ & $11,80$ & $1016,95$ & $141,6$\\
                \hline
                $14,00$ & $13,70$ & $1021,90$ & $191,8$\\
                \hline
                $16,00$ & $15,70$ & $1019,11$ & $251,2$\\
                \hline
                $18,00$ & $17,70$ & $1016,95$ & $318,6$\\
                \hline
                $20,00$ & $19,70$ & $1015,23$ & $394,0$\\
                \hline
            \end{tabular}
            \caption{Messwerte für den Widerstand}
            \label{tab:widerstand}
        \end{table}

        \begin{figure}[ht!]
            \centering
            
            \includegraphics[width=0.5\textwidth]{bilder/Physik_02.png}
            \caption{$U$ über $I$ Diagramm für Glühbirne (orange) und Widerstand (grau) \\ \textcolor{red}{Die X-Achse $\hat{=}$ Spannung in V und die Y-Achse $\hat{=}$ Stromstärke in A}}
            \label{fig:wertev1}
        \end{figure}
\newpage
        \begin{figure}[ht!]
            \centering
            \includegraphics[width=0.5\textwidth]{bilder/Physik_03.png}
            \caption{$R$ über $P$ Diagramm für Glühbirne (orange) und Widerstand (blau) \\ \textcolor{red}{Die X-Achse $\hat{=}$ Leistung in mW und die Y-Achse $\hat{=}$ Widerstand in $\Omega$}}
            \label{fig:wertev2}
        \end{figure}

      Die Steigung einer Gerade, hier von der Grauen aus dem Schaubild \ref{fig:wertev1}, lässt sich mithilfe der Formel \ref{eq:steigung} berechnen. Dabei entspricht die Steigung R dem Widerstand. $\Delta x$ entspricht der Spannung in Volt und $\Delta y$ der Stromstärke in Ampere. 
      
     \begin{center}
     	\begin{equation}
      	\label{eq:steigung}
      	R = \frac{\Delta y}{\Delta x}
      \end{equation}
     \end{center}
     
     Wählt man nun geschickt $\Delta x= 5V$  $\Delta y=5A$ so erhält man den Widerstand $R = 1 \Omega$

    Man kann aus den Diagrammen so wie den berechneten und gemessenen Werten noch weitere Erkenntnisse ziehen. Zum einen sieht man das im Diagramm \ref{fig:wertev1} die steigende Spannung keinen Einfluss auf den Widerstand hat, während der Widerstand bei der Glühbirne steigt. Dies liegt daran, dass die Glühbirne sich stark erwärmt (Glühdraht) und wenn sich Metalle erwärmen die Atome in Schwingungen geraten was wiederum zu mehr Kollisionen der Elektronen zur Folge hat. Und da mehr Kollision = höhere Widerstand steigt der Widerstand der Glühbirne (Ohmsches Gesetz). 
    
    Beim Widerstand sollten sich die Werte nicht verändern. In Tabelle \ref{tab:widerstand} tut er es aber trotzdem. Dies kommt durch die Messfehler beim abnehmen der Werte zustande. 

Im Diagramm \ref{fig:wertev2} sieht man ganz gut das mit steigendem Widerstand die Glühbirne weniger Leistung verheizt wie der Widerstand mit konstantem Widerstand. Dies liegt daran, dass die Glühbirne mit steigender Spannung weniger Gesamtrom fließen hat.
\section{Versuch 2: Schaltkreise mit Widerständen}

    \subsection{Versuchsaufbau und -durchführung}

        In Versuch 2 wird folgender Schaltkreis verwendet:
\newpage
			\begin{center}
				 \includegraphics[width=0.5\textwidth]{bilder/Physik_04.png}
				\captionof{figure}{Aufbau Versuch 2}
				\label{fig:versuch2}
			\end{center}
           

		\vspace{1em}
        Zunächst wird eine Spannung $U_{\mathrm{ges}} = 5\ \mathrm{V}$ an die Schaltung angelegt. Nun werden die Teilspannung $U_{3}$ und der Teilstrom $I_{2}$ gemessen und anhand dieser Messungen die restlichen Werte berechnet. Die Messwerte werden in Tabelle \ref{tab:versuch2} festgehalten.
        Danach wird der Versuch wiederholt, jedoch wird die Spannung so gewählt, das eine Stromstärke von $I_{mathrm{ges}} = 10\ \mathrm{mA}$ fließt. Die Messwerte werden ebenfalls in Tabelle \ref{tab:versuch2} festgehalten.

    \subsection{Ergebnisse}

        \begin{table}[h!]
            \centering
            \begin{tabular}{|l|l|l|l|l|l|l|l|l|l|l|}
                \hline
                $I_{\mathrm{ges}}$ in $\mathrm{mA}$ & $I_{1}$ in $\mathrm{mA}$ & $I_{2}$ in $\mathrm{mA}$ & $I_{3}$ in $\mathrm{mA}$ & $U_{\mathrm{ges}}$ in $\mathrm{V}$ & $U_{1}$ in $\mathrm{V}$ & $U_{2}$ in $\mathrm{V}$ & $U_{3}$ in $\mathrm{V}$ & $R_{1}$ in $\mathrm{\Omega}$ & $R_{2}$ in $\mathrm{\Omega}$ & $R_{3}$ in $\mathrm{\Omega}$\\
                \hline\hline
                $3,81$ & $2,41$ & $1,40$ & $3,81$ & $5,00$ & $0,60$ & $0,60$ & $3,80$ & $248,96$ & $428,57$ & $997,38$\\
                \hline
                $10,00$ & $6,10$ & $3,90$ & $10,00$ & $13,40$ & $1,63$ & $1,63$ & $10,14$ & $267,21$ & $417,95$ & $1014,00$\\
                \hline
            \end{tabular}
            \caption{Messwerte für Versuch 2}
            \label{tab:versuch2}
        \end{table}

        Die Berechnung der Werte erfolgt bei beiden Messungen nach folgendem Schema. Zunächst lässt sich $I_{3}$ berechnen, da die Stromstärke in einer Reihenschaltung überall gleich ist und $I_{1}$, da sich die Stromstärke in Parallelschaltungen aufteilt und $I_{\mathrm{ges}}$ und $I_{2}$ gegeben sind.
        Danach lassen sich die Spannung $U_{1}$ und $U_{2}$ berechnen, da sich die beiden Widerstände als ganzes betrachtet in Reihe geschaltet sind und sich die Spannung auf jedes Element in einer Reihenschaltung aufteilt. Und in Parallelschaltungen ist die Spannung überall gleich.
        Zuletzt lassen sich die jeweiligen Widerstände mit der Formel berechnen:

        \begin{equation}
            R = \frac{U}{I}
            \label{eq:widerstand}
        \end{equation}

        Wir nehmen an, dass die Temperatur der Leiter konstant bleibt und so das Ohmsche Gesetz angewendet werden kann. Nun können wir den zweiten Versuchsteil erklären. Die Stromstärke wird auf 10mA festgesetzt. Mit dem Ohmschen Gesetz folgt das die Stromstärke konstant bleibt, unabhängig vom Messpunkt im Stromkreis. Wir wissen auch das sich die Stromstärke in einer Parallelschaltung abhängig vom Widerstand aufteilt. Da wir $I_2$ gemessen haben können wir $I_3$ einfach mit der Formel $I_3 = I_{ges} - I_2$ bestimmen. Da wir $U_3$ messen können wir mit der Formel \ref{eq:widerstand} nun auch $R_3$ bestimmen. Aus der Gesamtspannung $U_{ges}$ die uns das Netzteil anzeigt wissen wir das sich eine Teilspannung von 3.62 V auf der Parallelschaltung befindet. Da sich Spannung darin gleichmäßig aufteilt (Kirchoffsches Gesetz) können wir $U_1$ und $U_2$ leicht berechnen. Nun können wir jeweils wieder \ref{eq:widerstand} anwenden und die anderen 2 Widerstände berechnen.
        
        Für den ersten Versuchsteil nehmen wir ebenfalls an das die Temperatur der Leiter konstant bleiben um das Ohmsche Gesetz anzuwenden. Auch hier wissen wir das $I_{ges}$ konstant bleibt. Daraus können wir mit $U_3$ und der Formel \ref{eq:widerstand} $R_3$ berechnen. Da sich Spannung in einer Parallelschaltung gleichmäßig aufteilt können wir wieder leicht $U_1$ und $U_2$ bestimmen. Und mit $I_2$ dann auch leicht $I_1$. 
        
        Diese Messungen sind natürlich auch fehlerbehaftet da unsere Geräte Temperaturschwankungen unterlegen sind. Allerdings sind die Meßfehler hier zu vernachlässigen.
        \newpage

\section{Versuch 3: Wechselspannung und Oszilloskop}

    \subsection{Versuchsaufbau und -durchführung}
        
        In Versuch 3 wird mit einem Signalgenerator eine sinusförmige Wechselspannung mit $f = 40\ \mathrm{kHz}$ erzeugt. Diese soll mit dem Oszilloskop beobachtet werden. Anschließend soll eine Schirmskizze angefertigt werden und ausgewertet werden. Die Auswertung ist in Tabelle \ref{tab:versuch3} festgehalten. Als zweite Spannung wir eine rechteckförmige Wechselspannung mit $f = 2,5\ \mathrm{kHz}$ erzeugt. Die Auswertung ist ebenfalls in Tabelle \ref{tab:versuch3} festgehalten.
        Abbildungen \ref{fig:versuch3sinus} und \ref{fig:versuch3eckig} zeigen die Schirmskizzen.

    \subsection{Ergebnisse}

       	\begin{center}
       	\includegraphics[width=0.5\textwidth]{bilder/Sinus.png}
       	\captionof{figure}{Aufbau Versuch 2}
       	\label{fig:versuch3sinus}
       \end{center}
		Die Amplitude lässt sich leicht ablesen, da jedes Kästchen 2 Volt in Y-Richtung und 5 $\mu s$ in X -Richtung entspricht. Daraus folgt eine Amplitude von 6.2V und eine Periode von 25.0 $\mu s$
	\begin{center}
	\includegraphics[width=0.5\textwidth]{bilder/Eckig.png}
	\captionof{figure}{Aufbau Versuch 2}
	\label{fig:versuch3eckig}
\end{center}
Die Amplitude lässt sich leicht ablesen, da jedes Kästchen 1 Volt in Y-Richtung und 100 $\mu s$ in X -Richtung entspricht und. Daraus folgt eine Amplitude von 2.0V und eine Periode von 400.0 $\mu s$
        \begin{table}[h!]
            \centering
            \begin{tabular}{|l|l|l|l|}
                \hline
                $f_{\mathrm{geg}}$ in $\mathrm{kHz}$ & Amplitude in $\mathrm{V}$ & Periode $T$ in $\mathrm{\mu s}$ & Frequenz $f$ in $\mathrm{kHz}$\\
                \hline\hline
                $40,0$ & $6,2 \pm 0,4$ & $25,0 \pm 1,0$ & $40,0 \pm 1,6$\\
                \hline
                $2,5$ & $2,0 \pm 0,2$ & $400,0 \pm 20,0$ & $2,5 \pm 0,125$\\
                \hline
            \end{tabular}
            \caption{Auswertung für Versuch 3}
            \label{tab:versuch3}
        \end{table}
\newpage
        Die Fehlerberechnung ergibt sich aus dem Größtfehler:

        \begin{equation}
            \Delta f = \frac{1}{T^2} \cdot \Delta \mathcal{T}
            \label{eq:größtfehler}
        \end{equation}

        mit $\Delta \mathcal{T} = 1\ \mathrm{\mu s}$, bzw. $\Delta \mathcal{T} = 20\ \mathrm{\mu s}$. Die Amplitude besitzt einen Fehler von $0,4\ \mathrm{V}$, bzw. $0,2\ \mathrm{V}$. Der Fehler der Amplitude ist allerdings nicht für die Berechnung der Frequenz relevant.

        Dies sind offensichtlich Fehlerbehaftete Größen. Für einen Fehler in X-Richtung von 0.4 V bei \ref{fig:versuch3sinus} und 0.2 V bei \ref{fig:versuch3eckig} haben wir uns entschieden dies die kleinste Ablesbare Einheit ist. Den Fehler für die Zeit haben wir analog dazu festgelegt. Der Fehler für die Frequenz wurde durch die Genauigkeit des Oszilloskops festgelegt.

\section{Versuch 4: Ladeverhalten eines Kondensators}

    \subsection{Versuchsaufbau und -durchführung}

        \begin{figure}[H]
        	\begin{subfigure}{0.48\linewidth}
        		\includegraphics[width=\linewidth]{bilder/versucha.png}
        		\caption{Bild 1}
        		\label{fig:bild1}
        	\end{subfigure}
        	\hfill
        	\begin{subfigure}{0.48\linewidth}
        		\includegraphics[width=\linewidth]{bilder/versuchb.png}
        		\caption{Bild 2}
        		\label{fig:bild2}
        	\end{subfigure}
        	\caption{Versuchsaufbau Versuch 4}
        	\label{fig:zwei-bilder}
        \end{figure}
        Ohmscher Widerstand $R = 100\ \mathrm{\Omega}$ für sowohl den Aufbau \ref{fig:bild1} als auch \ref{fig:bild2}. Die Frequenz wird auf $f \approx 250Hz $ festgelegt
\newpage
    \subsection{Ergebnisse}
        Die entstehenden Schirmbilder sehen wie folgt aus:
        
        \begin{figure}[H]
        	\begin{subfigure}{0.48\linewidth}
        		\includegraphics[width=\linewidth]{bilder/teila.png}
        		\caption{Schirmbild von Aufbau a)}
        		\label{fig:biold4}
        	\end{subfigure}
        	\hfill
        	\begin{subfigure}{0.48\linewidth}
        		\includegraphics[width=\linewidth]{bilder/teilb.png}
        		\caption{Schirmbild von Aufbau b)}
        		\label{fig:bild3}
        	\end{subfigure}
        	\caption{Schirmbilder vom jeweiligen Aufbau}
        	\label{fig:zwei-bilder-schirme}
        \end{figure}
        
        \begin{table}[h!]
            \centering
            \begin{tabular}{|l|l|l|l|}
                \hline
                $n$ & Zeit $t$ in $\mathrm{ms}$ & Amplitude in $V$ & $I$ in $A$\\
                \hline\hline
                $1$ & $0,000$ & $6,48$ & $0,0648$\\
                \hline
                $2$ & $0,023$ & $4,08$ & $0,0408$\\
                \hline
                $3$ & $0,048$ & $2,56$ & $0,0256$\\
                \hline
                $4$ & $0,073$ & $1,36$ & $0,0136$\\
                \hline
                $5$ & $0,098$ & $0,96$ & $0,0096$\\
                \hline
                $6$ & $0,123$ & $0,64$ & $0,0064$\\
                \hline
                $7$ & $0,148$ & $0,40$ & $0,0040$\\
                \hline
                $8$ & $0,173$ & $0,32$ & $0,0032$\\
                \hline
            \end{tabular}
            \caption{Messwerte für Versuch 4}
            \label{tab:versuch4}
        \end{table}

        \begin{figure}[h!]
            \centering
            \includegraphics[width=0.5\textwidth]{bilder/Physik_05.png}
            \caption{Ladekurve für Versuch 4 \\ \textcolor{red}{In Y-Richtung Spannung in Volt und in X-Richtung Zeit in ms}}
            \label{fig:versuch4}
        \end{figure}

       Was bei den Schirmbildern in Figur \ref{fig:zwei-bilder-schirme} gleich auffällt ist, dass wenn man genau Hinschaut Bild b) das Komplement von Bild a) ist. Warum das so ist erklären wir gleich. 
       
       Wir möchten nun die Kapazität $C$ vom Kondensator berechnen. Dazu müssen wir folgende Annahmen treffen. Der Stromkreis ist verlustfrei, das heißt es geht nichts durch Wärme verloren. Dann gilt das Ohmsche Gesetz. Weiter können wir mit der Maschenregel $U_1,\ U_2,\ U_3$ berechnen. Denn entlang einer Masche ist die Summe der Spannungen Null. Das heißt $-U_1 + U_2 + U_3 = 0V  \Rightarrow U_2 + U_3 = U_1$. Anhand von der Maschenregel erklärt sich auch warum das Schirmbild \ref{fig:biold4} das Komplement zu Schirmbild \ref{fig:bild3} ist.
       
       Wir können nun die Stromstärke anhand von der Formel:

       \begin{equation}
       	I(t)=I_0e^{\frac{-t}{\tau}} \text{wobei für $\tau$ gilt: } \tau = R \cdot C
       	\label{eq:kapazität}
       \end{equation}