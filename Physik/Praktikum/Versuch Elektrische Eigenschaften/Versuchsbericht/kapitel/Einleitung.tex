\chapter*{1 Einleitung}
\addcontentsline{toc}{chapter}{1 Einleitung}
\setcounter{chapter}{1}
\setcounter{section}{0}
\setcounter{subsection}{0}

In diesem Versuchsprotokoll steht die vertiefte Analyse elektrischer Eigenschaften im Fokus, welche als fundamentale Grundlage für eine Vielzahl moderner Technologien und Anwendungen dienen. Die eingehende Untersuchung gliedert sich in vier wesentliche Versuche, die sich den Themengebieten der Strom-Spannung Kennlinien, Schaltkreise mit Widerständen, Wechselspannung und Oszilloskop sowie dem Ladeverhalten eines Kondensators widmen. Jeder dieser Versuche eröffnet ein Fenster in die komplexe Welt elektrischer Phänomene, ermöglicht eine präzise Analyse der Wechselwirkungen zwischen Strom und Spannung und gewährt vertiefte Einblicke in die Charakteristiken elektrischer Bauelemente.

Die praxisnahe Erkundung dieser Themen nicht nur vertieft theoretisches Wissen, sondern ermöglicht auch das Verständnis der fundamentalen Prinzipien, welche die Grundlage elektrischer Systeme bilden. Durch die systematische Annäherung an die untersuchten Aspekte streben wir an, nicht nur die theoretischen Konzepte zu internalisieren, sondern auch deren praktische Anwendungen in unserem täglichen Leben zu begreifen. Dieses Protokoll bietet somit eine eingehende Analyse der elektrischen Eigenschaften, welche weitreichende Erkenntnisse und eine erweiterte Perspektive auf die Bedeutung dieser Phänomene in unserer technologisch geprägten Umgebung ermöglichen wird.