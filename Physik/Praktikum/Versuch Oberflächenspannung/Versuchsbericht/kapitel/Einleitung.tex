\chapter*{1 Einleitung}
\addcontentsline{toc}{chapter}{1 Einleitung}
\setcounter{chapter}{1}
\setcounter{section}{0}
\setcounter{subsection}{0}

Im heutigen Versuch untersuchen wir die Oberflächenspannung. Dieser Begriff beschreibt die unterschiedlichen Anziehungskräfte von Molekülen und fasst diese in messbare Größen zusammen. Bei Flüssigkeiten kann man das besonders gut messen da mit die Reibung vernachlässigen kann. In diesem Versuch betrachten wir die Abreißmethode sowie die Kapillarmethode. Bei der Abreißmethode wird eine Flüssigkeit aus einem Gefäß gezogen und die Kraft gemessen die dafür aufgewendet werden muss. Bei der Kapillarmethode wird eine Flüssigkeit in ein Gefäß mit einem dünnen Rohr gegeben und die Höhe der Flüssigkeit im Rohr gemessen.