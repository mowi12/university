\chapter*{1 Einleitung}
\addcontentsline{toc}{chapter}{1 Einleitung}
\setcounter{chapter}{1}
\setcounter{section}{0}
\setcounter{subsection}{0}

Die Untersuchung der Viskosität von Flüssigkeiten ist von entscheidender Bedeutung für zahlreiche Anwendungen in der Chemie, Physik und Ingenieurwissenschaften. Viskosität bezeichnet dabei die innere Reibungskraft, die einer Flüssigkeit entgegenwirkt, wenn sie in Schichten unterschiedlicher Geschwindigkeit gleitet. Diese Materialeigenschaft beeinflusst maßgeblich den Fluss von Flüssigkeiten und spielt eine essenzielle Rolle in verschiedenen technologischen Prozessen, angefangen von der Schmierung von Maschinen bis hin zur Gestaltung effizienter Transportmittel.

Das vorliegende Versuchsprotokoll widmet sich der Bestimmung der Viskosität von Getriebeöl mittels der Kugelfallmethode sowie der Messung der Glycerinkonzentration mithilfe des Kapillarviskosimeters. Die gewählten Methoden bieten nicht nur Einblicke in die grundlegenden Prinzipien der Viskosität, sondern bedienen sich auch etablierter physikalischer Gesetzmäßigkeiten, wie der laminaren Strömung, der Stokes'schen Reibungskraft, des Kräftegleichgewichts, der Strömung in einem Kapillarrohr sowie des Hagen-Poiseuille'schen Durchflussgesetzes.

Die Kugelfallmethode, basierend auf den Prinzipien der laminaren Strömung und der Stokes'schen Reibungskraft, ermöglicht die Bestimmung der Viskosität eines Fluids durch die Beobachtung des Sinkverhaltens einer Kugel in der Flüssigkeit. Dieser Ansatz erlaubt Rückschlüsse auf die innere Reibung und die Fließeigenschaften des Getriebeöls.

Das Kapillarviskosimeter hingegen nutzt das Hagen-Poiseuille'sche Durchflussgesetz, um die Viskosität von Flüssigkeiten durch ihre Strömung in einem Kapillarrohr zu bestimmen. In diesem Experiment liegt der Fokus auf der Messung der Glycerinkonzentration, wobei die Viskosität als maßgeblicher Indikator für die Konzentration herangezogen wird.