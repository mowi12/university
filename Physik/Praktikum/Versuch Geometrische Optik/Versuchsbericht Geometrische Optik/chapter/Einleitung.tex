\chapter*{1 Einleitung}
\addcontentsline{toc}{chapter}{1 Einleitung}
\setcounter{chapter}{1}
\setcounter{section}{0}
\setcounter{subsection}{0}

Die Geometrische Optik ist ein grundlegender Bereich der Physik, der sich mit der Untersuchung des Lichts und dessen Verhalten bei der Ausbreitung in verschiedenen Medien befasst. Sie bildet die Basis für das Verständnis von Lichtquellen, Linsen, Spiegeln und anderen optischen Instrumenten. Durch die Annahme, dass Lichtstrahlen sich geradlinig ausbreiten und Wechselwirkungen an Oberflächen gemäß den Gesetzen der Reflexion und Brechung erfolgen, ermöglicht die Geometrische Optik eine anschauliche Darstellung und Beschreibung des Lichtverhaltens.

Das Ziel dieses Versuchsberichts ist es, die grundlegenden Prinzipien der Geometrischen Optik zu untersuchen und anhand von Experimenten das Verhalten von Lichtstrahlen beim Durchgang durch verschiedene optische Elemente zu analysieren. Dazu werden wir uns auf die Reflexion und Brechung von Lichtstrahlen an ebenen Spiegeln und transparenten Medien konzentrieren und deren charakteristische Eigenschaften sowie mathematischen Zusammenhänge untersuchen.

Durch die Durchführung verschiedener Versuche und die Auswertung der Ergebnisse werden wir die Gesetzmäßigkeiten der Geometrischen Optik empirisch überprüfen und eine Verbindung zwischen den theoretischen Konzepten und deren praktischer Anwendung herstellen. Ein vertieftes Verständnis dieser Prinzipien ist von entscheidender Bedeutung für zahlreiche Anwendungen in der Optik, darunter die Entwicklung von optischen Instrumenten, medizinischen Geräten und modernen Kommunikationstechnologien.

Der Bericht wird den experimentellen Aufbau, die verwendeten Methoden sowie die erhaltenen Ergebnisse und Schlussfolgerungen detailliert darlegen, um ein umfassendes Verständnis der Geometrischen Optik zu vermitteln und deren Bedeutung in verschiedenen technologischen und wissenschaftlichen Anwendungen zu unterstreichen.