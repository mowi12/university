\chapter*{1 Einleitung}
\addcontentsline{toc}{chapter}{1 Einleitung}
\setcounter{chapter}{1}
\setcounter{section}{0}
\setcounter{subsection}{0}

Die Charakterisierung der Oberflächenspannung von Flüssigkeiten, insbesondere des Wassers, stellt eine bedeutende Aufgabe in der physikalischen Chemie dar. Die Oberflächenspannung ist ein essenzielles Maß für die Fähigkeit einer Flüssigkeit, ihre Oberfläche zu minimieren und spielt eine entscheidende Rolle in zahlreichen naturwissenschaftlichen Phänomenen. In diesem Versuchsprotokoll konzentrieren wir uns auf die präzise Bestimmung der Oberflächenspannung von Wasser unter verschiedenen experimentellen Bedingungen.

Die Abreißmethode, bei der ein Ring aus der Flüssigkeit gezogen wird, ermöglicht es, die Kraft zu messen, die erforderlich ist, um zusätzliche Oberfläche in Form einer zylinderförmigen Flüssigkeitslamelle zu erzeugen. Durch diese Methode lassen sich präzise Informationen über die Oberflächenspannung gewinnen. Zudem wird die Steighöhe des Wassers in Kapillaren unterschiedlicher Dicke als alternative Methode zur Bestimmung der Oberflächenspannung herangezogen.

Ein weiterer Fokus dieses Protokolls liegt auf dem Einfluss von Substanzen, insbesondere Ionen und Tensiden, auf die Oberflächenspannung des Wassers. Diese Untersuchungen werden mithilfe der Abreißmethode durchgeführt, um Erkenntnisse über die Wechselwirkungen zwischen den Molekülen und der Oberfläche der Flüssigkeit zu gewinnen.

Darüber hinaus analysieren wir das Benetzungsverhalten von Wasser auf verschiedenen Oberflächen, indem wir den Kontaktwinkel bestimmen. Diese Untersuchung ermöglicht es, die Wechselwirkungen zwischen der Flüssigkeit und der Oberfläche zu charakterisieren, was für verschiedene Anwendungen in den Materialwissenschaften und der Oberflächenchemie von hoher Relevanz ist. Durch die Zusammenführung dieser experimentellen Ansätze streben wir eine umfassende und präzise Charakterisierung der Oberflächenspannung des Wassers und ihrer Einflussfaktoren an.